\documentclass{article}

% Language setting
% Replace `english' with e.g. `spanish' to change the document language
% \usepackage[english]{babel}
\usepackage[utf8]{inputenc}
\usepackage[russian]{babel}
\usepackage{enumitem}
\usepackage{graphicx}
\usepackage{array}
\usepackage{hyperref}
\usepackage[left=15mm, right=15mm]{geometry}

% Set page size and margins
% Replace `letterpaper' with`a4paper' for UK/EU standard size
\usepackage[letterpaper,top=2cm,bottom=2cm,left=3cm,right=3cm,marginparwidth=1.75cm]{geometry}

% Useful packages
\usepackage{amsmath}
\usepackage{graphicx}
\usepackage[colorlinks=true, allcolors=blue]{hyperref}

\title{Задание по алгоритму имитации отжига}
\author{Шутков Геннадий}

\begin{document}
\maketitle

\section{Формальная постановка задачи}

\subsection{Основные понятия}

\emph{Расписание} - упорядочение всех работ из множества $\{w_i\}_{i=1}^{N}$ по процессорам из множества $\{p_j\}_{j=1}^{M}$. Где каждой работе сопоставляется процессор, и момент начала выполнения работы на нем.
\\
\emph{Длительность расписания} - момент времени $T = \max(t_i)$, где $t_i$ - момент времени завершения последней работы на процессоре i.

\subsection{Формальная постановка}


\textbf{Дано}: 

\begin{itemize}
    \item Набор работ из N элементов.
    \item Набор процессоров из M элементов.
    \item Для каждой работы, время ее выполнения на процессоре.
\end{itemize}

\vspace{0.5cm}

\textbf{Требуется}: Построить расписание выполнения работ на процессорах.\\
То есть, определить привязку каждой работы к процессору и порядок выполнения работ на каждом из процессоров.

\vspace{0.5cm}

\textbf{Минимизируемый критерий}: Длительность расписания, т.е. время завершения последней работы.


\end{document}